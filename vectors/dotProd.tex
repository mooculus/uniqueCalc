\begin{document}


\begin{question}
  \[
  \vector{1,1} \cdot \vector{-1,1} = \answer{0}
  \]
  \begin{hint}
    By drawing these vectors, you can see that they are perpendicular,
    so according to property $5$, we must have that $\vector{1,1}
    \cdot \vector{-1,1} = 0$
  \end{hint}
\end{question}

\begin{question}
\[
\vector{2,3,1}\cdot\vector{2,3,1}= \answer{14} 
\]	

\begin{hint}
  By property $4$, we have that the dot product of any vector with
  itself is the square of its length.  The length of $\vector{2,3,1}$
  is $\sqrt{2^2+3^2+1^2}$, so the answer must be $2^2+3^2+1^2 =14$
\end{hint}
\end{question}

\begin{question}
  \[
  \vector{2,3}  \cdot \vector{4,-5}= \answer{-7} 
  \]
  This is a hard problem, and will require significant creativity to answer.
  
  \begin{hint}
    The only kinds of vectors we actually know how to compute the
    cross products of are vectors which are perpendicular or vectors
    which are parallel.  These vectors are neither.  However, we can
    break these vectors down into components which are all mutually
    either perpendicular or parallel!  In particular, the unit vectors
    $\mathbf{i} = \vector{1,0}$ and $\mathbf{j} = \vector{0,1}$ are
    perpendicular.
  \end{hint}
  
  \begin{hint}
    We can rewrite the product as	
    \[
    (2\mathbf{i}+3\mathbf{j}) \cdot (5\mathbf{i} +-5\mathbf{j})
    \]
  \end{hint}
  
  \begin{hint}
    \begin{align*}
      (2\mathbf{i}+3\mathbf{j}) \cdot (4\mathbf{i} +-5\mathbf{j}) &= 4(2\mathbf{i}+3\mathbf{j}) \cdot \mathbf{i}+ -5(2\mathbf{i}+3\mathbf{j}) \mathbf{j}\\
      &=8\mathbf{i} \cdot \mathbf{i} +12 \mathbf{i} \cdot \mathbf{j}-10\mathbf{i} \cdot \mathbf{j}-15\mathbf{j} \cdot \mathbf{j}\\
      &=8(1)+12(0)-10(0)-15(1)\\
      &=-7
    \end{align*}
  \end{hint}
\end{question}

The last question should convince you of the following theorem

\end{document}
