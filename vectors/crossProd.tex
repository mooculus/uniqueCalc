

\begin{question}
  If $\det(\vec{a},\vec{b},\vec{c}) = 12$, what is
  $\det(\vec{b},\vec{a},\vec{c})$?
  \[
  \det(\vec{b},\vec{a},\vec{c}) = \answer{-12}
  \]
  \begin{hint}
    \[
    \begin{vmatrix} 
      b_1 & a_1 & c_1\\
      b_2 & a_2 & c_2\\
      b_3 & a_3 & c_3\\
    \end{vmatrix}
    =b_1a_2c_3+a_2c_3b_3+c_1b_2a_3-b_3a_2c_1-a_3c_3b_1-c_3b_2a_1
    \]
    
    But this (make sure you check!) equal to
    $-\det(\vec{a},\vec{b},\vec{c})$.  Thus the answer is $-12$.
  \end{hint}
\end{question}

\begin{question}
  \[
  \det(\vec{a},\vec{a},\vec{b}) = \answer{0}
  \]
  \begin{hint}
    If you write out the definition, you will see that all the terms cancel, and you get $0$
  \end{hint}
\end{question}
	
\begin{question}
  Assume $\det(\vec{a},\vec{b},\vec{c}) = 3$.  Then 
  \[
  \det(5\vec{a},\vec{b},2\vec{c}) = \answer{30}
  \]
  \begin{hint}
    Since each term has an entry from $\vec{a},\vec{b}$ and $\vec{c}$,
    and each entry of each vector is getting multiplied by the same
    constant, we get an extra factor of $10$ in each term.  Thus the
    answer is $30$.
  \end{hint}
\end{question}

\begin{question}
  Assume $\det(\vec{a},\vec{b},\vec{c}) = 3$ and
  $\det(\vec{a},\vec{b},\vec{d}) = 4$.
  \[
  \det(\vec{a},\vec{b},\vec{c}+\vec{d}) = \answer{7}
  \]
  
  \begin{hint}
    Writing out the definition, and distributing, you will see that
    this is equal to the sum of the two original determinants.  So the
    answer is $7$.
  \end{hint}
\end{question}

The last few exercises strongly suggest the following theorem (which you have essentially already proven by doing the exercises above):
	
\begin{theorem}
  The determinant enjoys the following properties, and is in fact the
  \textbf{only} function enjoying such properties:
  
  \begin{itemize}
  \item Alternating: Switching any pair of entries in the determinant
    switches the sign.  For example $\det(\vec{a},\vec{b},\vec{c}) =
    -\det(\vec{c},\vec{b},\vec{a})$.
  \item Factor out scalars: Multiplying any entry by a constant scales
    the determinant by that constant.  For example
    $\det(k\vec{a},\vec{b},\vec{c}) = k\det(\vec{a},\vec{b},\vec{c})$
  \item Respect addition: For example
    $\det(\vec{a},\vec{b}+\vec{d},\vec{c}) =
    \det(\vec{a},\vec{b},\vec{c}) +\det(\vec{a},\vec{d},\vec{c}) $
  \item $\det(\vec{i},\vec{j},\vec{k}) = 1 $
  \end{itemize}
\end{theorem}

\begin{observation}
  By the alternating property, if the same vector appears more than
  once as an argument in a determinant, then the determinant must be
  zero.  For example consider $\det(\vec{a},\vec{a},\vec{b})$. By the
  alternating property we must have
  \[
  \det(\vec{a},\vec{a},\vec{b}) = -\det(\vec{a},\vec{a},\vec{b})
  \]
  but this implies 
  
  \[
  \det(\vec{a},\vec{a},\vec{b})=0
  \]
\end{observation}
	
We will not prove that the determinant is the only function with these
properties, but that is an important point.  If you ever wondered
where this crazy formula came from, this explains it.  If you want
these $4$ nice looking properties, there is only one function which
does it and it is this one.  You should be able to prove that
yourself, using just the properties.  Just start with a general
$\det(\vec{a},\vec{b},\vec{c}) $, and use the properties to reduce it
to sums of determinants involving only $\vec{i},\vec{j}$ and
$\vec{k}$.  The formula for the determinant will pop out.
	
It turns out that the determinant has the following geometric
interpretation:
\begin{theorem}
  If $\vec{a},\vec{b},\vec{c} \in \R^3$, then they span a
  parallelepiped. The volume of this parallelepiped is
  $\left|\det(\vec{a},\vec{b},\vec{c})\right|$. If this parallelepiped
  is not degenerate (has nonzero volume), then the sign of
  $\det(\vec{a},\vec{b},\vec{c})$ tells you whether the trio of
  vectors is \dfn{positively oriented} or \dfn{negatively oriented}
  (this is the definition of positively and negatively oriented).
\end{theorem} 
	
\begin{question}
  Let $\vec{a} = \vector{1,1,1}$, $\vec{b} = \vector{1,0,1}$ and
  $\vec{c} = \vector{1,0,0}$.  Is $(\vec{a},\vec{b},\vec{c})$
  positively or negatively oriented?
  \begin{multipleChoice}
    \choice[correct]{positively}
    \choice{negatively}
  \end{multipleChoice}
		
  What is the volume of the parallelepiped spanned by these three vectors?
  
  \[
  \textrm{Volume} = \answer{1}
  \]
  
  \begin{hint}
    \begin{align*}
      \begin{vmatrix}
	1 & 1 & 1\\
	1 & 0 & 0\\
	1 & 1 & 0
      \end{vmatrix} &=
      1(0)(0)+1(0)(1)+1(1)(1)-1(0)(1)-1(1)(1)-0(1)(1)\\
      &=1
    \end{align*}
    
    Thus this trio is positively oriented, and its volume is $1$.
  \end{hint}
\end{question}

\begin{question}
  Do the vectors $\vector{1,2,2}$, $\vector{3,4,1}$ and $\vector{5,8,5}$ all lie in the same plane?
  
  \begin{multipleChoice}
    \choice{No}
    \choice[correct]{Yes}
  \end{multipleChoice}
  
  \begin{hint}
    They will lie in the same plane if and only if the parallelepiped they span
    is degenerate, aka its volume is $0$.
  \end{hint}
  
  \begin{hint}
    So we just need to see whether the determinant of these three
    vectors is zero or not.
  \end{hint}
  
  \begin{hint}
    \begin{align*}
      \begin{vmatrix}
	1&3&5\\
	2&4&8\\
	2&1&5
      \end{vmatrix} &= 1(4)(5)+3(8)(2)+5(2)(1)-2(4)(5)-1(8)(1)-5(2)(3)\\
      &=20+48+10-40-8-30\\
      &=78-78\\
      &=0
    \end{align*}
    
    Thus the vectors must all lie in the same plane.  In fact, we can
    see that $2\vector{1,2,2}+\vector{3,4,1}=\vector{5,8,5}$, which
    confirms this fact.
  \end{hint}
  
\end{question}
We will not explicitly prove this theorem here.  However, you
should be able to convince yourself that the oriented volume
function $\textrm{Vol}(\vec{a},\vec{b},\vec{c})$ satisfies all
$4$ conditions of the determinant above.  Since there is only
one function satisfying these properties, the volume function
must be given by the determinant.
Here is a nice geometric interpretation of orientation.

\begin{theorem}
  Consider the trio of vectors $(\vec{a},\vec{b},\vec{c})$.  Assume
  that they form a nondegenerate parallelepiped.  Then the vectors
  $\vec{a}$ and $\vec{b}$ define a plane.  There are two sides to this
  plane.  If you take your right hand, and curl your fingers from
  $\vec{a}$ to $\vec{b}$, your thumb will be pointing to one side of
  the plane.  If $\vec{c}$ is on this side, then the trio is
  positively oriented.  Otherwise it is negatively oriented.
  \begin{center}
    BADBAD
    %\includegraphics[width=3 in]{RHR.jpg}
    All vectors on the same side of the plane that the thumb is pointing are positively oriented with respect to $(A,B)$
  \end{center}
\end{theorem}








