The simplest kind of vector valued function $\R \to \R^3$ is a
\textit{linear} function.

\begin{definition}
  A \dfn{linear} function from $\R \to \R^3$ is a function of the form
  \[
  \boldsymbol{\l}(t) = t\vec{v}
  \] 
  where $\vec{v} \in \R^3$ is a given vector.
\end{definition} 
 
\begin{question}
  If $f$ is a linear function, and $f(2) = \vector{2,4,6}$, what is
  $f(3)$?
  \[	
  f(3) = \vector{\answer{3},\answer{6},\answer{9}}
  \]
\end{question}
 
\begin{question}
  Assume $f:\R \to \R^3$ is a linear function.  Then 
  \[
  f(0) = \vector{\answer{0},\answer{0},\answer{0}}
  \]
\end{question}

Linear functions are called ``linear'' because they parameterize lines
through the origin.

 %BADBAD picture
 
 \begin{question}
   Compare and contrast the linear function $\vec{f}(t) = (1,2,3)$ and $\vec{g}(t)=(2,4,6)$	
   \begin{multipleChoice}
     \choice{They parameterize different lines}
     \choice{They parameterize the same line, but $\vec{f}(t)$ moves ``twice as fast'' as $\vec{g}(t)$ }
     \choice[correct]{They parameterize the same line, but $\vec{g}(t)$ moves ``twice as fast'' as $\vec{f}(t)$ }
     \choice{These are the same function!}
	\end{multipleChoice}
 \end{question}
 
 What if we want to parameterize the lines going through a different
 point $P$, instead of lines going through the origin?
 
 All we have to do is translate!

 \begin{definition}
   An \dfn{affine} function $f:\R \to \R^3$ is a function of the form
   \[
   f(t) = \vec{P}+t\vec{v}
   \]
   where $\vec{P}$ and $\vec{v}$ are vectors in $\R^3$
 \end{definition}
 
 An affine function parameterizes a line passing through the point
 $\vec{P}$, pointing in the direction $\vec{v}$
 
\begin{question}
  If $f(t) = \vector{x(t),y(t),z(t)}$ is an affine function and passes
  through the points $f(0) = \vector{1,2,3}$ and $f(1) =
  \vector{2,2,2}$, then
  \begin{align*}
    x(t) &= \answer{1+t}\\
    y(t) &= \answer{2}\\
    z(t) &= \answer{3-t}
  \end{align*}
  \begin{hint}
    Let $f(t) = \vec{P}+t\vec{v}$.  Then $f(0) = \vec{P} = \vector{1,2,3}$
  \end{hint}
  \begin{hint}
    \begin{align*}
      f(1) &= \vector{2,2,2}\\
      \vector{1,2,3}+1\vec{v} &= \vector{2,2,2}\\
      \vec{v} = \vector{1,0,-1}
    \end{align*}
  \end{hint}
  \begin{hint}
    Thus $f(t) = \vector{1,2,3}+t\vector{1,0,-1} = \vector{1+t,2,3-t}$
  \end{hint}
\end{question}

\begin{explanation}
  We can use affine functions to parameterize any line in space, but
  as we have already seen, sometimes you can find more than one affine
  function which parameterizes a given line (some may ``move faster''
  than others).
  
  Often, for a given line, we will already know that the line passes
  through a point $P$, and points in a direction $\vec{v}$.  Then
  finding an affine function parameterizing the line is easy: we can
  just take $f(t) = \vec{P}+t\vec{v}$.
  
  If we know that a line passes through two points $P$ and $Q$, then
  we know that it points in the direction $\vec{v} = \vec{Q} -
  \vec{P}$, and passes through $P$.  So we can parameterize it as
  $f(t) = \vec{P}+t(\vec{Q} - \vec{P})$.
\end{explanation}

\begin{question}
  Using the idea above, find an affine function parameterizing the
  line passing through $\vec{P} = \vector{0,2,4}$ and $\vec{Q} =
  \vector{1,1,1}$.
  \[
  f(t) = \vector{\answer{t},\answer{2-t},\answer{4-3t}}
  \]
  \begin{hint}
    The line passes through $\vec{P}$ and points in the direction
    $\vec{Q} - \vec{P} = \vector{1,1,1} - \vector{0,2,4} =
    \vector{1,-1,-3}$.
  \end{hint}
  \begin{hint}
    Thus the line is parameterized by 
    \[
    f(t) = \vector{0,2,4}+t\vector{1,-1,-3}
    \]
  \end{hint}
\end{question}
