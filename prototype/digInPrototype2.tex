\documentclass{ximera}

\newcommand{\RR}{\mathbb R}
\renewcommand{\d}{\,d}
\newcommand{\dd}[2][]{\frac{d #1}{d #2}}
\renewcommand{\l}{\ell}
\newcommand{\ddx}{\frac{d}{dx}}
\newcommand{\dfn}{\textbf}
\newcommand{\eval}[1]{\bigg[ #1 \bigg]}


\title[Dig-In:]{Differentiability implies continuity}

\begin{document}
\begin{abstract}
We see that if a function is differentiable at a point, then it must
be continuous at that point.
\end{abstract}
\maketitle

There is some connection to continuity and differentiability.

\begin{theorem}[Differentiability Implies Continuity]
If $f(x)$ is a differentiable function at $x = a$, then $f(x)$ is
continuous at $x=a$.
\end{theorem}

To explain why this is true, we are going to use the following
definition of the derivative
\[
f'(a) = \lim_{x\to a} \frac{f(x)-f(a)}{x-a}.
\]
\begin{proof}
Assuming that $f'(a)$ exists, we want to show that $f(x)$ is
continuous at $x=a$, hence we must show that
\[
\lim_{x\to a} f(x) = f(a).
\]
Consider
\begin{align*}
\lim_{x\to a} \left(f(x) - f(a)\right) &= \lim_{x\to a} \left((x-a)\frac{f(x) - f(a)}{x-a}\right) &\text{Multiply and divide by $(x-a)$.} \\
&= \left(\lim_{x\to a} (x-a) \right) \left(\lim_{x\to a}\frac{f(x) - f(a)}{x-a}\right) &\text{Limit Law.} \\
&= 0\cdot f'(a) = 0.
\end{align*}
Since 
\[
\lim_{x\to a}\left(f(x) - f(a)\right) = 0 
\]
we see that $\lim_{x\to a} f(x) = f(a)$, and so $f(x)$ is continuous.
\end{proof}

This theorem is often written as its contrapositive:
\begin{quote}
If $f(x)$ is not continuous at $x=a$, then $f(x)$ is not
differentiable at $x=a$.
\end{quote}


Thus from the theorem above, we see that all differentiable functions
on $\RR$ are continuous on $\RR$. Nevertheless there are continuous
functions on $\RR$ that are not differentiable on $\RR$.

INTERACTIVE?

\end{document}
