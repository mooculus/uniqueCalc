\documentclass{ximera}


\begin{document}
Differential calculus is the study of the rate of change of a function.  In this activity, we introduce this idea through an example.

\section{Average Velocity}

First we review the concept of the average velocity of an object.

\begin{question}
	A car is moving at $40 \textrm{kph}$ due north,  and at noon it is $20$ kilometers north of Columbus.  At $2:30 \textrm{pm}$ it will be \answer{120} kilometers north of Columbus.
\end{question}

\begin{question}
	A car is traveling west.  At $1 \textrm{pm}$ it is $30$ kilometers west of Cleveland.  At $3:00 \textrm{pm}$ it is $130$ kilometers west of Cleveland.  The average velocity of the car over this time span is \answer{50} $\textrm{kph}$. 
\end{question}

\begin{question}
	\begin{hint}
		\begin{question}
			At $2:00 \textrm{pm}$ the car is \answer{40} kilometers south of Cincinnati.
		\end{question}
		\begin{question}
			At $4:00 \textrm{pm}$ the car is \answer{160} kilometers south of Cincinnati.
		\end{question}
		\begin{question}
			So in $2$ hours, the car traveled \answer{120} kilometers.
		\end{question}
		\begin{question}
			Thus its average velocity over this time is \answer{60} kilometers per hour.
\end{question}
		
	\end{hint}
	A car is traveling south.  At a time $t$ hours after noon, the car has traveled $10x^2$ kilometers south of Cincinnati.  The average velocity of the car from $2:00 \textrm{pm}$ to $4:00 \textrm{pm}$ is \answer{60} kilometers per hours.
\end{question}

\section{Instantaneous velocity }

The average velocity of a function only makes sense over an interval of time.  Nevertheless, we often talk about our speed at a particular instant, as in the sentence ``At $5:24 \textrm{pm}$ I was driving $32 \textrm{kph}$''.  Probably you would only know that if you happend to look at your speedometer at the time that the clock changed to $5:24$ on your dashboard.

Although we use this kind of expression all the time, it is not at all clear exactly what we mean by it.  Imagine we were accelerating at $5:24 \textrm{pm}$, so that at $5:23 \textrm{pm}$ I was going $30 \textrm{kph}$ and at $5:25 \textrm{pm}$ I was going $35 \textrm{kph}$.  Then I was only going $32 \textrm{kph}$ for an instant! In any given instant, no time elapses, and no distance is traveled.  So what on Earth could be meant by the speed at an instant?  This is a deep philosophical problem, and one that \href{wiki}{quite puzzled the ancient greeks}.

\begin{question}

What do you think about this paradox?  Do you think you know what ``At $5:24 \textrm{pm}$ I was driving $32 \textrm{kph}$'' means?  Try to explain it plain language below.
\begin{free-response}

\end{free-response}

\end{question}

Now we will try to subject this kind of question to a more rigorous analysis.

A car is driving on a highway.  Assume that $t$ is the time in hours, and $x(t)$ is the position of the car (in kilometers) at time $t$.  Assume that $x(t)$ can be modeled by the function $x(t) = 10t^2$.  We want to figure out what the speed of the car at the time $t=1$ is.

\begin{question}
	The average velocity of the car from the time $t=1$ to the time $t=2$ is \answer{}
\end{question}

\begin{question}
	The average velocity of the car from the time $t=1$ to the time $t=1.1$ is \answer{}
\end{question}

\begin{question}
	The average velocity of the car from the time $t=1$ to the time $t=1.01$ is \answer{}
\end{question}

\begin{question}
	These average velocities appear to be getting closer to the number \answer{}
\end{question}

	Let us make this rigorous with our knowledge of limits.  

\begin{question}
	The average velocity of the car from the time $t=1$ to the time $t=1+h$ is $Avg(h)=$ \answer{}
\end{question}

\begin{question}
	The limit of the average velocity as $h \to 0$ is $\lim_{h \to 0} Avg(h) =$\answer{}
\end{question}

Intuitively, as we measure the average velocity over a smaller and smaller interval of time, we should be getting closer and closer to the instantaneous velocity.  We now \textbf{define} the instantaneous velocity to be a limit of average velocities over smaller and smaller intervals.  (Depending on your philosophical outlook, this could be viewed as a cop out, but this is a math class so definitions rule our world). 

\begin{question}
	Let  $x(t)$ be the position of an object at time $t$.  We define its instantaneous rate of change, $x'(t)$, at time $t=a$ to be the limit of its average velocity over the interval $[a,a+h]$ as $h \to 0$.  Symbolically,
	
	%Need to be able to declare some variable as a function, and then test expression against a few ``random'' functions.
	\[
	x'(a) = \lim_{h \to 0} \answer{(x(a+h)-x(a))/h}
	\]
	
	We also call this quantity ``The \textbf{derivative} of $x$ with respect to $t$ at the time $t=a$''
\end{question}

The study of derivatives will occupy much of the rest of this course.

\end{document}