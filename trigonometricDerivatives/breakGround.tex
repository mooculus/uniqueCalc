\documentclass{ximera}

\newcommand{\RR}{\mathbb R}
\renewcommand{\d}{\,d}
\newcommand{\dd}[2][]{\frac{d #1}{d #2}}
\renewcommand{\l}{\ell}
\newcommand{\ddx}{\frac{d}{dx}}
\newcommand{\dfn}{\textbf}
\newcommand{\eval}[1]{\bigg[ #1 \bigg]}


\outcome{Compute derivatives of trigonometric functions.}
\outcome{Understand the cyclic nature of the derivatives of sine and cosine.}


\title[Break-Ground:]{Sine is the key}

\begin{document}
\begin{abstract}
Two young mathematicians think about derivatives of trigonometric functions.
\end{abstract}
\maketitle

Check out this dialogue between two calculus students (based on a true
story):

\begin{dialogue}
\item[Devyn] Riley! I think we know more derivatives than we think we know.
\item[Riley] Really? You must tell me what you are thinking.
\item[Devyn] Well, it seems to me that almost all trig functions are
  basically built from sine. For instance, I know that 
  \[
  \cos(\theta)  = \sin(\theta + \pi/2).
  \]
\item[Riley] Continue.
\item[Devyn] Ah. So, by the chain rule
  \begin{align*}
  \dd{\theta}\cos(\theta) &= \dd{\theta} \sin(\theta + \pi/2)\\
  &=\cos(\theta + \pi/2).
  \end{align*}
\item[Riley] Right on!
\item[Devyn] Hmmmm. But it is also true that
  \[
  \cos(\theta) = \sin(\pi/2 - \theta), 
  \]
  and now by the chain rule
  \begin{align*}
  \dd{\theta}\cos(\theta) &= \dd{\theta}\sin(\pi/2-\theta)\\
  &=-\cos(\pi/2 - \theta).
  \end{align*}
\item[Riley] Wow!
\item[Devyn] This is awesome. We are going to ask so many questions in
  math class today!
\end{dialogue}

Here we have a mystery! It is true that:
\[
\cos(\theta)  = \sin(\theta + \pi/2) \qquad\text{and}\qquad\cos(\theta) = \sin(\pi/2 - \theta).
\]
However, using these facts we find two different answers for the derivative of $\cos(\theta)$, namely
\[
\dd{\theta} \cos(\theta) = \sin(\theta + \pi/2) \qquad\text{and}\qquad\dd{\theta}\cos(\theta) =-\cos(\pi/2 - \theta).
\]

\begin{problem}
  Check all of the following that are correct.
  \begin{selectAll}
    \choice[correct]{$\dd{\theta} \cos(\theta) = \sin(\theta + \pi/2)$}
    \choice[correct]{$\dd{\theta}\cos(\theta) =-\cos(\pi/2 - \theta)$}
    \choice{None of the above.}
  \end{selectAll}
\end{problem}

\begin{problem}
   In your own words, explain why $\cos(\theta) = \sin(\theta + \pi/2)$.
   \begin{freeResponse}
   	One way to think about this relationship is to recall that the graph
	of the cosine function is the same as the graph of the sine function,
	except shifted to the left by $\pi/2$.
   \end{freeResponse}
\end{problem}

\input{../leveledQuestions.tex}

\end{document}
